%%This is a very basic article template.
%%There is just one section and two subsections.
\documentclass[a4paper,12pt]{article}

  %\usepackage[ngerman]{babel}   % deutsche Sprachanpassung
  \usepackage[latin1]{inputenc} % erlaubt Umlaute in der tex-Datei f�r Windows
  %\usepackage[applemac]{inputenc}	% erlaubt Umlaute in der tex-Datei f�r Mac
  \usepackage{ucs}				% advanced support for using UTF-8
  %\usepackage[utf8x]{inputenc}	% erlaubt Umlaute in der tex-Datei mit ucs
  % (fuktioniert nict)
  \usepackage[T1]{fontenc}      % Trennung bei W�rtern mit Umlauten
  %\usepackage[DIV15]{typearea}  % Vergr��erung des Textbereichs
  \usepackage{amsmath}			% /equref und anders
  % (http://www.ams.org/publications/authors/tex/amslatex)
  \usepackage{amsthm}			% f�r Satz-, Beweis-, Lemma- Umgebung (\newtherom)
  \usepackage{amssymb}			% mehr Symbole
  \usepackage{graphicx}			% einbinden von Grafiken
  \usepackage{textcomp}       	% Zus�tzliche Symbolzeichen
  \usepackage[nottoc,numbib]{tocbibind}	% "`Literatur"' erscheint im
  % Inhaltsverzeichniss
  \usepackage{stmaryrd}			% f�r Blitz bei Wiederspruch
  \usepackage{pdfpages}			% Einbinden von pdf / Vektorgrafiken
  % Pdf einf�gen mit \includepdf[pages=1-4]{Meindoku.pdf}
  \usepackage[pdftex]{hyperref}	% have hyperref package before float in order to
  % get strange errors with .\theHfloatbox, Au�erdem Hyperlink f�r \ref
  \usepackage{float}			% f�r Figure Haarlinie
  \floatstyle{boxed} 			% f�r Figure Haarlinie
  \restylefloat{figure}			% f�r Figure Haarlinie
  \usepackage{framed, color}	% f�r grau hinterlegten Text
  \definecolor{shadecolor}{gray}{.9}	% f�r grau hinterlegten Text
  
\setlength{\marginparwidth}{2,8cm} %alles folgende f�r Randnotizen
\usepackage{marginnote}
\usepackage{tikz}
\let\oldmarginnote\marginnote
\renewcommand\marginnote[1]{\-\oldmarginnote%
  {
    \begin{tikzpicture}[remember picture]%
      \definecolor{margincolor}{rgb}{0.9,0.9,0.9}% draw=black
      \draw node[fill=margincolor, text width = \marginparwidth] (inNote)%
      {\RaggedRight\footnotesize #1};%
    \end{tikzpicture}%
  }
}

\begin{document}


\section{Error Model for Mass Spectrometry}

\subsection{Context}

In order to understand the Nf-$\kappa$B signaling pathway of the cell a good
model of the process has to be obatined. The state of the art aproche is used
which is to define a structure of the model with certain parameters left to be
fitted such that the model agrees with the messured data. This
structure is generated by pure thought and prior knowleg of the systems. A vital
part of this pursuite is to determine the functional dependencies of values in the model and
there degree of freedome represented by the number of parameters that can adjust
these dependencies. It is of high interesst to limit this number of
paramters to a minimum that just allows the model to fit all realistic systems
of the kind it is suppose to model but not more, as it could get ``overfitted''
which results in unrealistic predictions.

Since messured data is not free of noise this noise has to be accounted for when
trying to fit the model. Thus the model should not predict the messured data
exactly but the messured data should be destributed arround the prediction just
as the messurment is distributed arround the real values due to the noise of
messurment. Since these destributions are not know they have to be fittet as
part of the model. The right type of destribution has to be picked and the
parameters fittet such that it represents the destribution of the messurements
arround the prediction and therefor hopefully the real value.

With the error model obtained we can calculate the probability of a certain
messurent to be taken and hence the porobability to messure a set of
certain values like the messurments already taken. This liklyhood to messure the existing data
 will be used as the goodness of fit. Thus fitting the selected model means
 trying to find a set of parameters for the model that maximizes this
 likelyhood.

\subsection{Goal}

Here we want to analyze the mechanics of the procedure to messure the data in
order to obtain an understanding of the noise which is produced relative to the
real values. We hope to derive a general functional depedency of the messured
values to their errors with only a few paramters to be fitted.
Experirience in this field of science has shown that such mechanistic approches
ussually do not work perfectly due to the lack of accurate knowlege about the
underlying process. Henc alternative modles will be worked out and compared by
means of ACI, AIC$_C$, BIC, etc.
If sufficent data is provided we will also compute Shapiro-Wilk tests and the
like to analyse the distribution of error arround an established prediction,
e.g. multiple messurments of the same dilution of a certain protein. If ther is
more than one of such data sets we could also analyze the errors dependency to
properties of the protein as size or lipophilicity.

\subsection{Mechanic Description}

The cells which ought to be analyzed has a certain volume $V_C$ and conatins a
descrete number $N_C$ of the protein of interest. This number can also be
represented by the concentraion of the protein in the cell $C_C$ by
\[
N_C = V_C \cdot C_C
\]
This cell lives in a culture of $n$ mostly equal cells which is lysated to
aquire the lysate with volume $V_L$, number of proteins $N_L = N_C \cdot n$ and
an concentration $C_L = \frac{N_L}{V_L}$ which is the average of the protein
concentrations of the individual cells of the culture. In the analyzis of a
constant state this averaging can be an advantage as it corrects for errors due
to biological diversity and absorbs some deviations of outlyers. But in the
observation of a dynamic time dependent process it could be source of a major
falsification of the data. The stimulus given to the culture does not reach each
cell at an equal momemt of time and the cells probably do not react equally
resulting in a divers set of reaction curves with peaks at different times and
maybe even completly different shapes. An averaging at each individual point of
time over all the cells would result in a reaction curve that is not
representative for any typical cell reaction. Another uncertenty is weather
intercellular communictaion can result in long forced delays of the muessured
process in some cells. However the resulting deviation of the messured curve to
the true cell respons is hard to compute and has to be ignored for now. As the
process which is to be analyzed does only occure over several hours and is
messured in timestepps of 30 minutes to 2 hours we hope be mimally effected by
the diversity of respons in time.

However all the folling steps to generate the resulting messurment can be
modeled in a lot easyer fassion and will be adressed now in a much more
satisfying manner.
As the protein travels from the lysate through the filters, ionisation, splicing
and finally will be detected by a fraction of focus it has probabilty of $1-p$
to get lost and a probability $p$ to be detected.


\end{document}
